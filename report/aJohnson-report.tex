% Report for MATH6641 Project
% Copyright (C) 2018 Andrew Johnson, GTRC
% This program is free software: you can redistribute it and/or modify
% it under the terms of the GNU General Public License as published by
% the Free Software Foundation, either version 3 of the License, or
% (at your option) any later version.
% 
% This program is distributed in the hope that it will be useful,
% but WITHOUT ANY WARRANTY; without even the implied warranty of
% MERCHANTABILITY or FITNESS FOR A PARTICULAR PURPOSE.  See the
% GNU General Public License for more details.
% 
% You should have received a copy of the GNU General Public License
% along with this program.  If not, see <http://www.gnu.org/licenses/>.
\documentclass{article}
\newcommand{\pdiff}[2]{\frac{\partial#1}{\partial#2}}
\newcommand{\ddx}[1]{\pdiff{#1}{x}}
\newcommand{\ddt}[1]{\pdiff{#1}{t}}
\newcommand{\dpmdx}{\ddx{\psi_m}}
\newcommand{\dpmdt}{\ddt{\psi_m}}
\newcommand{\fisTrms}{\chi\nu\sigma_f}
\newcommand{\tfor}{\text{ for }}
\usepackage{amsmath}
\title{$FeS_NT$ - A Finite Element Neutron Transport Solver}
\author{Andrew Johnson}
\usepackage[left=1in,right=1in,top=0.75in,bottom=1in]{geometry}
\usepackage{cleveref}

\begin{document}

\maketitle

\section{Introduction}
The goal of this work is to introduce and explain the code written for the MATH6641 project.
The time-dependent one-dimensional neutron transport equation was solved using the discrete 
ordinates formulation \textbf{cite} using discontinuous-Galerkin in space, and
explicit Euler for time.
The project was written in Python and works best with 3.5\footnote{not tested on other versions}
and is included with this report in \texttt{fesnt.py}.
This report, and the code included, are covered with the GNU Public License, found in \texttt{LICENSE}.

\subsection{Problem Statement}
The discrete-ordinates formulation of the neutron transport equation chooses to solve for the angular
neutron flux, $\psi(x, \mu, t)$ along a discrete set of angles $\vec{\mu}$. A quadrature set of 
weights and angles is chosen to satisfy some physical constraints, briefly summarized here:
\begin{subequations}
    \begin{equation}
        \mu_m = -\mu_{N-m+1}
    \end{equation}
    \begin{equation}
        \sum_m w_m = 2
    \end{equation}
    \begin{equation}
        \int_{-1}^1f(x, \mu)d\mu\approx\sum_mw_m f(x, \mu_m)
    \end{equation}
\end{subequations}
This project seeks to solve the following initial-boundary value problem:
\begin{equation} \label{eq:ibvpNTE}
    \frac{1}{\nu}\dpmdt+\mu_m\dpmdx+\sigma_t(x)\phi(x, t) = \frac{1}{2}\left(\sigma_{s0}+\fisTrms\right)
    \sum_mw_m\psi_m(x, t), \tfor a<x<b,t>0, \mu_m\in\vec{\mu}
\end{equation}
subject to initial and boundary conditions
\begin{subequations}
    \begin{equation}
        \psi_m(x, 0) = \psi_{m,0}(x), \tfor a<x<b, \mu_m\in\vec{\mu}
    \end{equation}
    \begin{equation}
        \psi_m(a, t) = \Gamma_a(\mu_m, t)\tfor \mu_m\in\vec{\mu},t>0
    \end{equation}
    \begin{equation}
        \psi_m(b, t) = \Gamma_b(\mu_m, t)\tfor\mu_m\in\vec{\mu}, t>0
    \end{equation}
\end{subequations}
\section{Derivation of the Method} \label{sec:derive}

\section{Implementation} \label{sec:implement}

\subsection{Using \texttt{fesnt.py}} \label{sec:usage}

\section{Results} \label{sec:results}

\subsection{Case 1: Pure-Absorber} \label{sec:pa}

\subsection{Case 2: Non-fissile} \label{sec:nonFissile}

\subsection{Case 3: Fissile Problem} \label{sec:fissile}

\section{Conclusion} 

\end{document}

